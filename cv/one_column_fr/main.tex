\documentclass[a4paper,12pt]{article}

\usepackage{url}
\usepackage{parskip} 	
\RequirePackage{color}
\RequirePackage{graphicx}
\usepackage[usenames,dvipsnames]{xcolor}
\usepackage[scale=0.9]{geometry}
\usepackage{tabularx}
\usepackage{enumitem}
\newcolumntype{C}{>{\centering\arraybackslash}X} 
\usepackage{supertabular}
\usepackage{tabularx}
\newlength{\fullcollw}
\setlength{\fullcollw}{0.47\textwidth}
\usepackage{titlesec}				
\usepackage{multicol}
\usepackage{multirow}

\titleformat{\section}{\Large\scshape\raggedright}{}{0em}{}[\titlerule]
\titlespacing{\section}{0pt}{10pt}{10pt}

\usepackage[style=authoryear,sorting=ynt, maxbibnames=2]{biblatex}

\usepackage[unicode, draft=false]{hyperref}
\definecolor{linkcolour}{rgb}{0,0.2,0.6}
\hypersetup{colorlinks,breaklinks,urlcolor=linkcolour,linkcolor=linkcolour}
\addbibresource{citations.bib}
\setlength\bibitemsep{1em}

\usepackage{fontawesome5}

\newenvironment{jobshort}[2]
    {
    \begin{tabularx}{\linewidth}{@{}l X r@{}}
    \textbf{#1} & \hfill &  #2 \\[3.75pt]
    \end{tabularx}
    }
    {

    }

\newenvironment{joblong}[2]
    {
    \begin{tabularx}{\linewidth}{@{}l X r@{}}
    \textbf{#1} & \hfill &  #2 \\[3.75pt]
    \end{tabularx}
    \begin{minipage}[t]{\linewidth}
    \begin{itemize}[nosep,after=\strut, leftmargin=1em, itemsep=3pt,label=--]
    }
    {
    \end{itemize}
    \end{minipage}    
    }



\begin{document}
\pagestyle{empty} 

\begin{tabularx}{\linewidth}{@{} C @{}}
\Huge{Augustin Bresset} \\[7.5pt]
\href{https://github.com/augustin-bresset}{\raisebox{-0.05\height}\faGithub\ augustin-bresset} \ $|$ \ 
\href{https://www.linkedin.com/in/augustin-bresset}{\raisebox{-0.05\height}\faLinkedin\ Augustin Bresset} \ $|$ \ 
\href{mailto:augustin.bresset@ip-paris.fr}{\raisebox{-0.05\height}\faEnvelope \ augustin.bresset@ip-paris.fr} \ $|$ \ 
\href{tel:+33651092688}{\raisebox{-0.05\height}\faMobile \ +33 6 51 09 26 88}  \\
\end{tabularx}

\section{Résumé}
Passionné par la recherche à l'intersection des \textbf{mathématiques} et de l'\textbf{informatique}, je suis à la recherche d'un stage de recherche.
Actuellement en double diplôme : \textbf{Master Data Science à l'École Polytechnique} et cycle ingénieur à \textbf{Télécom SudParis}.
Souhaite poursuivre un \textbf{doctorat} dans le domaine des sciences des données.

\section{Expérience professionnelle}

\begin{joblong}{Ingénieur Logiciel Stagiaire - Rubicon (Bangkok)}{Mai 2025 - Août 2025 \textit{(Césure)}}
\item Développeur chargé de migrer l'ERP de l'entreprise (40 employés) depuis zéro sur \textbf{Odoo 18 Community}.
\item Conception de \textbf{schémas de base de données}  et développement de \textbf{10+ modules personnalisés} pour les prix, la production et la gestion documentaire.
\item Mise en place de \textbf{pipelines d'import de données} automatisés pour migrer des bases de données volumineuses (200K+ enregistrements).
\end{joblong}

\begin{joblong}{Stagiaire Recherche - ENSTA Paris (Palaiseau, France)}{Sept 2024 - Fév 2025 \textit{(Césure)}}
\item Conception d'un prototype de \textbf{framework Python} unifiant la gestion de jeux de données robotiques hétérogènes (\textbf{ROS bags}, \textbf{format KITTI}, etc.).
\item Travail avec \textbf{ROS} et \textbf{PyTorch} pour le prétraitement et l'évaluation de modèles dans l'environnement de recherche du laboratoire.
\end{joblong}

\begin{joblong}{Stagiaire Logiciel - Upfund (Paris, France)}{Juin 2023 - Août 2023}
\item \textbf{Upfund} combine données géospatiales et IA pour aider les acteurs du retail à trouver des emplacements optimaux.
\item Développement d'un \textbf{assistant cartographique IA} utilisant \textbf{Django}, \textbf{Angular}, l'\textbf{API OpenAI GPT} et \textbf{OpenStreetMap}, permettant des requêtes complexes basées sur la localisation.
\item Présentation d'un prototype fonctionnel aux clients en fin de stage.
\end{joblong}

\section{Formation}

\begin{joblong}{Master Data Science à \textbf{l'École Polytechnique}}{2025 - 2026}
    \item Cours de mathématiques : Statistique en grande dimension, Chaînes de Markov Cachées et méthodes Monte Carlo séquentielles, Analyse convexe, Optimisation pour le ML, Approximation stochastique
    \item Cours d'informatique : Deep Learning, Reinforcement Learning, Data Stream, Online Learning
\end{joblong}

\begin{joblong}{Cycle Ingénieur en Informatique - Télécom SudParis}{2022 - 2026}
\item Parcours orienté Statistiques, Machine Learning et Data Engineering.
\item Cours pertinents : \textbf{Processus Stochastiques}, \textbf{Théorie des valeurs extrêmes}, \textbf{Réseaux de neurones}, \textbf{Apprentissage par renforcement} (GPA 4.0/4.0).
\end{joblong}

\begin{joblong}{Auditeur au Master MVA - ENS Paris-Saclay}{2024 - 2025}
    \item NPM3D : Nuages de points et modélisation (validé)
\end{joblong}

\begin{joblong}{Classe Préparatoire (MP, Informatique) au Lycée Saint-Louis}{2021 - 2022}
    \item Équivalent d'une double licence en mathématiques et physique, avec une option en informatique.
\end{joblong}

\section{Projets}

\begin{joblong}{Swarm Rescue Challenge 2025}{Janvier - Mars 2025}
\item Classé \textbf{5e sur 40+ équipes} lors d'un challenge de programmation organisé par \textbf{CIEDS} et \textbf{ENSTA Paris}.
\item Développement d'un \textbf{système multi-agents} en Python pour la coordination de drones en mission de recherche et sauvetage.
\item Conception axée sur la simplicité et la robustesse : \textbf{A*}, \textbf{Filtrage Gaussien}, \textbf{Filtre de Kalman}.
\end{joblong}

\begin{joblong}{CodinGame Bot Programming}{2021 - 2024}
\item Participation régulière aux concours en ligne sur \textbf{CodinGame}.
\item Classé dans le \textbf{Top 1\% mondial} sur le défi \textit{Ultimate Tic-Tac-Toe} grâce à \textbf{l'élagage Alpha-Beta}.
\end{joblong}

\begin{joblong}{Autotech - Club Robotique Intech' (Télécom SudParis)}{2022 - 2023}
\item Chef de projet pour la conception d'un véhicule autonome au sein du \textbf{Club Intech'}.
\item Encadrement d'une petite équipe et implémentation du système de contrôle via \textbf{ROS 1}, \textbf{C++}, \textbf{Python}.
\end{joblong}

\section{Activités Extra-scolaires}

\begin{joblong}{Chef de Projet - Autotech, Club Intech' (Télécom SudParis)}{2023 - 2024}
\item Pilotage de la continuation du projet de véhicule autonome.
\end{joblong}

\begin{joblong}{Trésorier - Association Welcom' International Students}{2023 - 2024}
\item Gestion du budget et organisation d'événements favorisant l'intégration des étudiants internationaux.
\end{joblong}

\section{Compétences}
\begin{tabularx}{\linewidth}{@{}l X@{}}
\textbf{Noyau} & Python, Django, Odoo, ROS 1 \& 2 \\
\textbf{Machine Learning \& IA} & PyTorch, TensorFlow, Ingénierie ML, Modélisation statistique, Algorithmes d'optimisation \\
\textbf{Données \& Systèmes} & Analyse \& Visualisation, Modélisation de bases (PostgreSQL), Pipelines de données, Linux \\
\textbf{Développement logiciel} & API (REST, GraphQL), Architecture logicielle, Git/GitHub, Bash, C++ \\
\textbf{Web} & Angular, HTML, CSS \\
\end{tabularx}

\section{Langues}
\begin{tabularx}{\linewidth}{@{}l X@{}}
Français & Langue maternelle \\
Anglais & C1/C2 (TOEIC 920)\\
\end{tabularx}

\vfill

\end{document}
